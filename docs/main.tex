\documentclass{article}
\usepackage[utf8]{inputenc}
\usepackage{amsmath}
\usepackage{graphicx}
\graphicspath{ {images/} }
\newcommand{\Mod}[1]{\ (\mathrm{mod}\ #1)}

\title{Zerocaf: Highly secure efficient scalar operations for Discrete log based proof systems}
\author{
  Carlos Perez\\
  Dusk Foundation\footnote{https://dusk.network/}\\
  \texttt{carlos@dusk.network}
  \and
  Luke Pearson\\
  Dusk Foundation\\
  \texttt{luke@dusk.network}
}
\date{October 2019}

\begin{document}



\maketitle
\thispagestyle{empty}
\pagestyle{empty}

\begin{abstract}
Zerocaf is one of the cryptographic protocols built by, and implemented within, the Dusk Network, which uses zero-knowledge proofs to show the existence of a private key within one of many public keys. For this, there will be an elliptic curve, used for the key generation, defined over a Ristretto scalar field which enables the use of Ristretto in Bulletproofs while simultaneously abstracting the computationally intensive conversion within Rank-1 Constraint System from cofactor 8 scalar field into a cofactor 1 Ristretto field. This paper provides an explanation of the current curve development, as well as a contextual understanding of how this curve implementation acts as one of aspects within the Zerocaf protocol. 


\end{abstract}

\newpage

\tableofcontents

\newpage

\section{Introduction}
The construction and use of elliptic curves is paramount to many cryptographic protocols. Elliptic curves are among the fastest performing primitives where the discrete logarithm problem is hard, which is why they are regarded dominant in the field of cryptography. As the field of cryptography advances, elliptic curves have been proved to be unparalleled in their use as a cryptographic system at which speed and security are two of the most outstanding features. The implementation of elliptic curves within Zerocaf was designed such that it meets end goals of a larger protocol base. The use of elliptic curves is ultimately for both private and public key generation, these keys are used in conjunction with zero-knowledge proofs to show how they relate to one another. When outlining elliptic curve implementations, there are a multitude of features which are affected depending on design; often regarded as factors of opportunity cost which leads to compromise on critical features within Elliptic Curve Cryptography (ECC).\\\\  Elliptic curves have both a base field, which is the finite field in which they are defined; as well as a scalar field, which is associated with the number of points on the curve. Using discrete log based proof systems for general computation requires both an elliptic curve and an arithmetic circuit. These circuit, in these cases, is defined over the scalar field of the curve and encodes a relation between the input and outputs - this gives greater choice to the broadness of circuit operations, for example to encode an ECDH key exchange for the curve with bulletproofs as the argument. For this encoding to be performed, there must be an outermost curve which implements the proof and an inner curve - across which the proofs are defined. The inner curve, often called the Embedded curve, refers to a curve which has a quasi-construction inside of another curve. When constructing   \\\\

As with all elliptic curves, their construction will strongly determine the outcomes of the protocols in which they are implemented. In addition to this, there can be a discrepancies in both the security and speed of cryptographic systems dependant on how they're implemented. For example, the fastest point addition laws for elliptic curves is for Twisted Edwards curves, and for in circuit operations Montgomery curves. Whilst these curves models can provide some of the fastest and most simplistic operations, they do provide issues in security. However, neither Montgomery, nor Edwards curves deliver prime order groups in their implementations. They provide curves which have a cofactor which multiplies the prime subgroup to give the group order. The mismatch in desirability of prime order group and inability to implement one directly from the curve can be patched with unique tailored modifications, however, these fixes oft become perplexing to the non-implementors. Using curves which provide prime order groups, such are Weierstrass Curves, have slower formulae and are very difficult to implement in constant time. \\ 

There exists multiple applicable cryptographic protocols to different elliptic curves, and often to different curves within one family of curves. Applying ad hoc corrections to redress the abstraction mismatches with individual protocols can be tedious, and lead to inefficient implementations of each curve model. As families of curves are connected to one another through curve mappings, also known as isogenies, then the properties of these mappings can be integrated with protocols to extend them to different curve models.




\subsection{Implementation considerations}
\\ \subsubsection{$Embedding$}
The discrete log based proof is based on a defined outer curve but the in circuit operations are based on the field elements which are modulo the scalar field, then the construction of an embedded curve which has this as a base field make the operations considerably more efficient. This is defined in the following manner: Let $\varepsilon_{1}$ and $\varepsilon_{2}$ be elliptic curves. Where the prime sub-group order, or scalar field, of $\varepsilon_{1}$ is $r$; we define $\varepsilon_{2}$ over the base field $F_p$, where \#$F_p$ = $r$.
This will allows us to make fast in circuit operations using $\varepsilon_{2}$, with  $\varepsilon_{1}$ as the 
\subsubsection{$Security$}  Regarded by many as the most significant factor determining how elliptic curves should be implemented - we present the amalgamation of several different ECC techniques, tailored to a family of curves, in order to provide high security for the operations which use the curve. The most prominent of these techniques is cofactor compression. This is performed using the neoteric Ristretto[], which is an extension of Decaf, used to construct prime order groups from non prime order elliptic curves. Ristretto is applicable to Edwards[] curves with a cofactor of 4 or 8, where it handles the mismatch between the point on an Edwards curve and representative point of degrees up to and equalling the cofactor. \\\\
 In theory, the difficulty of breaking cryptographic systems stems solely from the hardness of the mathematical problems on which they are based. However, this proves not to be the case in practical implementations because of side channel attacks, which target the implementation as medium of encoding the cryptography - to circumvent these attacks, the operations need to be performed in constant time. The use of Edwards curves results in a uniform implementation which better facilitates these constant time operations. The Edwards form of a curve is considered complete, as any two inputs, given as x and y, provide a correct result.   
\subsubsection{$Speed$} To ensure that the operations remain amongst the fastest of the generic ECC operations, the Edwards form of the curve is used. This is because the Edwards forms of the curves provide the fastest known formulas, which can be accredited to extended Twisted Edwards, introduced by Hysil et al, where auxiliary points are used with fewer field inversions. \\\\
Curve models provide circumstantial speeds, where the environment in which the curve operations are being performed dictates which curve form is best to use. To capitalise on this, several implementations for given curve parameters ought to be defined. 
\subsubsection{$Compatibility$} (isogenies/facility) ristretto - sig - scalar

\section{Notation and Formulae} 
\subsection{General notation}
$Finite Field$: $F_p$ is the finite field where the $char$ \neq {2}\vee{3}\\\\
$\varepsilon_{a,d}$ is a Edwards curve, given by the equation: $$ {a}x^2+y^2=1+{d}x^2y^2 $$ where {$d$} and {$ad$} are none square in $F$ and has no points at infinity. In this paper, the primary focus is upon Twisted Edwards curves, where $a = -1$. The identity point of an Edwards curve, $\varepsilon$, where (X,Y) $\epsilon$ in $F_p$, is given encoded to (0,1). When Edwards points are expressed in Extended Twisted coordinates, the identity encoding is given by (X : Y : Z : T) = (0 : 1 : 1 : 0).\\\\
${M}_{a,\frac{2-4d}{a}}$ is a Montgomery curve, given by the equation: $$ y^2=x^3+Ax^2+Bx $$ A Montgomery curve is Birationally equivalent to an Edwards curve - a definition used for algebraic substitution -  where its point at of infinity is the identity point, denoted as (0 : 1 : 0).\\\\  
$\jmath_{a^{2},a−2d}$ is a Jacobi curve, given by the equation: $$y
^2 = {e}x^4 + 2Ax^2 + 1$$ A Jacobi curve, better known as a Jacobi quartic, is central to all curve models and to utilise this curve relationship we will only be using Jacobi curves where $e = {a}^2$, as such curves have a full 2-torsion point.\\\\
$Torsion\ points$:\ An element [P] in  $G$ is a torsion point if there is a mapping of $M$, by means of multiplication, where $M$ \textbullet\ [P] = $0_{G}$. Torquing elements for a curve form a subgroup, $G$[$M$], where the order is divisible by ${M}^2$. The torsion subgroups for this curve family have order 1, 2 or 4.  \\\\ 
$Isogeny$: An isogeny, $\varphi$ , is a function which maps algebraic groups whilst preserving the group structure. This mapping must satisfy the properties of being surjective and having a finite kernel. The isogeny, in this paper, is used to transport an encoding between different curve models.\\\\
$Curve forms$: $\varepsilon_{a,d}$; ${M}_{a,\frac{2-4d}{a}}$; $\jmath_{a^{2},a-{2d}}$ \\\\ These curve models are all isogenous to one another. The Edwards, Montgomery and Twisted Edwards are independently 2-isogenous to the Jacobi quartic and are therefore all 4-isogenous to one another.  \\\\
$Arithmetic$ \ $circuits$: \\\\
$Cofactor compression$: This refers a quasi-construction of cofactor 1 curves from cofactor 8 groups. Also known as cofactor division, it involves the process of point compression when points of order 4 or 8 are produced.\\ 

\subsection{Our contributions}
\subsubsection{Elliptic curve}
Here we present an elliptic curve, created for an safe and efficient implementation of bulletproofs; called Sonny. Sonny is defined as an embedded curve which the gives the input for the proofs and the discrete log based proof is implemented using the outer curve, Curve25519. \\\\

\noindent\fbox{%
    \parbox{\textwidth}{%
    $$ Sonny  $$
        \begin{itemize}
    \item Curve equation in Twisted Edwards form: $$ ax^2+y^2=1-dx^2y^2 $$ 
    \item $a= -1$
    \item $d= -\frac{126296}{126297}$
    \item $Basepoint: Y = \frac{3}{5}$\\
    \item Montgomery form equivalent: $$ y^2=x^3+Ax^2+x $$
    \item $A = 505186 $
    \item $Basepoint: X = 4$\\
    \item The curve group order, G, is $$ 2^{252}+115924404605461509904689566245241897752 $$      
    \item The order of the scalar field, $r$, is $$ 2^{249}+15114490550575682688738086195780655237219 $$       \item The order of the base field, $p$, is  $$ 2^{252} + 27742317777372353535851937790883648493 $$
    \item Cofactor: $$ h =\frac{G}{r} = 8$$
    \item Weierstrass form equivalent: \\  $$y^2=x^3+ax+b $$
    \item $a$ = 7237005577332262213973186563042994240857116359379907606001\\950828033483786813
    \item $b$ = 445582015604702849664
\end{itemize}
    }%
}\\\\


\subsubsection{Implementation features}
\begin{itemize}
    \item As is previously explained our curve has a base field which is the scalar field of curve25519, to allow for the use of efficient zero knowledge on operations within a circuit. Typically, implementations are designed in constant time, as explained in $1.1.2$, to prevent side channels attacks. Whereas in circuit operations can greater benefit from variable time implementations, as they perform faster. They can be applied when there is no secret data to protect, as they may lead to leakage of data. We therefore present an implementation which performs proofs in constant time with high security, and  verification in variable time and high speed. \\ 
    \item In order to avoid the drawbacks that are tied to having a cofactor, and therefore non prime order curves, we apply the Ristretto technique. This protocol builds upon the Decaf paper, where a technique to canonically lift points is presented for cofactor 4 curves. With an just one additional step into the Decaf compression algorithm the protocol is extended to cofactor 8 curves, Ristretto concatenates this into the algorithm thus making a it applicable to Twisted Edwards curve models.\\
    \item By using the Ristretto technique, we are able to solve all cofactor related issues in one place and with one step. This is facilitated by its use in the relationship of the curves, and how this lets us transport the cofactor compression for curves, via the isogeny, to another curve in the same family. Which in turn means we work with prime order points in any operations of ECC. Otherwise, the implementation would have to deal with the issue at varying stages which is dependent upon a protocols ultimate design. \\
\end{itemize}
 
\section{Set Inclusion}
Set inclusion will be used to show that a private key is one of many public keys. The private and public keys are two generated values from a one way function which are used in a cryptographic system. In any such system, the public key is used for encryption whereas only the private key is used for decryption. A public key is classed as set element, where the set is all of the curve points generated by a base point. The use of set inclusion is to prove that the private key exists as one of many public keys. A set can qualify as a subset of the other set if and only if the elements of the former set are likewise present, yet not the sole elements of the latter set. In order to produce a set inclusion proof, the Prover $\mathcal{P}$ has to convince the Verifier $\mathcal{V}$ that a given set is a subset of another set.
\subsection{Example}
A simplistic example of the logic outlined above is demonstrated hereafter. If:  $$ A={1,3,5} $$
$$ B={1,5} $$ then \textbf{\textit{B}} is a subset, or \textbf{\textit{‘proper subset’ of A.}} 
It is also important to note that if: $$ B={1,3,5} $$ then \textbf{\textit{B} would not be a subset of \textit{A} as \textit{B = A}}, in this case. Also, if $$ B={1,4} $$ then \textbf{\textit{B} would be a subset but not a proper subset of \textit{A}}, as every element of a $B$ must simultaneously be part of $A$ for the subset to exist.
\subsection{Advantages}
\begin{itemize}
    \item The advantage of using subsets is that they have varying mathematical properties, the one which is most pertinent to us is the proof that a subset exists inside of a set.
    \item From this, operations can be performed to that particular subset which can be used to show properties and create proofs of the larger subset without the extra expense as the whole set is not being used.
\end{itemize}
A full comprehension of this subset rule is very helpful, as well as largely applicable to the defined curve. \\
For the current set inclusion use case, due to the set elements being public keys and the input being a private key, there needs to be a $ScalarBaseMult (P=x\cdot G)$ operation.
\section{Bulletproofs and Rank 1 Constraint Circuits} 
Bulletproofs[1] are short non-interactive zero-knowledge proofs[2]. For example, Bulletproofs can be used to prove that an encrypted number is in a given range, without leaking any information about the number. Compared to SNARKs$[2]$, Bulletproofs require no trusted setup, which further reduces the risk of a malicious set up. However, Bulletproofs verification is computationally more intensive relative to the SNARK proof verification. Bulletproofs, in context to their computational intensity, have linear scaling, which is measured as the size of the arithmetic circuit.\\
Bulletproofs are designed to enable efficient confidential transactions in Bitcoin and other cryptocurrencies. Every transaction contains a cryptographic proof which proves the validity of the spending transaction. Bulletproofs shrink the size of the cryptographic proofs from over 10kB to less than 1kB. To prevent overflows every confidential transaction must carry a proof that all amounts are positive and smaller than a threshold. Such range proofs are much smaller with Bulletproofs, this also allows for $m$ transactions to have valid range proofs. \\
Bulletproofs have many other applications in cryptographic protocols, such as shortening proofs of solvency, short verifiable shuffles, confidential smart contracts, and as a general drop-in replacement for Sigma-protocols. \\
Bulletproofs are an optimization to the \emph{Efficient Zero-Knowledge Arguments for
Arithmetic Circuits in the Discrete Log Setting} paper. The aforementioned paper introduced an inner-product argument by the following diagram.
\begin{center}
\includegraphics[width=8.5cm]{images/circ.png}
\end{center}
The constraint system has the following format: 
\begin{itemize}
    \item  An vector of $n$ multiplications that gives $3 \cdot n$ low-level variables: left, right and output
    \item An vector of $q$ linear constraints between these variables.
    \item Additional \emph{m high-level variables} that represent external facts.
\end{itemize} 
 Although Bulletproof implementation provides a solid means of creating fast proofs, the prior choice of curve is important to ensure that binary decomposition is not needed within the circuit for reduction. This reduction is negated as the curve is defined over the Ristretto scalar field.  

\section{The Ristretto Scalar Field}
Ristretto $[4]$ is a technique that constructs prime-order elliptic curve groups, the construction of these groups stems from non prime-order Elliptic curves. Ristretto builds upon the Decaf paper$[5]$, where prime-order curve groups are created from curves with co-factor 4. Ristretto, on the other hand, is applicable to Edwards curve groups which have a co-factor 4 or 8. Edwards curve have a point of order 4, this means that points on the curve are not of prime order and they instead have a small co-factor. By using the Ristretto technique the abstraction problem is solved for all potential co-factor related issues with a single protocol. For use of the Ristretto scalar field in this implementation, any chosen curve needs to be defined over the Ristretto scalar field, for the prime-order group Ristretto255. This Ristretto scalar field provides a prime-order group of size $2^{252}$ $[4]$ by encoding group elements. The ristretto255 group will be implemented using points from the curve defined in the next section.  This protocol compresses the co-factor of a curve, with the rationale of being able to avoid the drawbacks that are concurrent with a co-factor, whilst being able to capitalize on the robustness of an otherwise solid curve.\\ If a curve given in standard elliptic curve form, defined as: \\\\
\begin{itemize} 
    \item $Y = X^3 + Ax + B$\\\\
    then\\
    \item Let $G$ be a group of prime order for the curve, denoted as $q$
    \item  A co-factor, denoted by $h$, exists such that the order of the curve is $h \cdot q$ for the large prime $q$ 
\end{itemize} 
\hfill \break
There are various advantages and disadvantages to having a co-factor larger than one, therefore a thorough analysis must be performed,  so that it is known whether or not co-factor manipulation is needed. For all curves, except for Hessian curves, the co-factor is divisible by 4. To become more useful to a broad spectrum of cryptography, Ristretto is apt for a large number of curves, which have a co-factor of 8 or 4. When the co-factor is greater than 1 multiple operations can be hindered. In the case of set inclusion, having a co-factor larger than one will hinder the curve operations, specifically relating to the scalar base operations. In reference to the need for subset proofs, the goal is obstructed where the co-factor is not compressed, which leads to non-injective behaviour between the groups. Non-injective functions in set mappings, which is a method to describe whether an element exists in another set or not, affects the operations in proving subsets exists within sets. \\\\
For elliptic curves, any scalar multiplication is a 1 to 1 mapping if the group order is prime. Only in a prime-order group is a random scalar for the operation valid, and it must be in the range 1 to $q$-1. Whereas in a non prime-order group, the adding of a small element can lead to a small subgroup confinement attack$[6]$, which makes it possible to present the same result from different inputs. When implemented, Ristretto acts as a thin layer, which provides a protocol to construct a prime-order group.  \\\\
To embed a curve into this prime field, the definition that an embedded curve L, is a curve whose base field is defined by the scalar field of another curve, M. In this case, the Doppio curve, which will be eluded to shortly, has a base field which is equal to the scalar field defined by Ristretto255. To visualise how this protocol is performed, when the curve is embedded into the Ristretto scalar field - two arbitrary Edwards points, $P$ and $Q$, may be represented as the equivalent Ristretto points in the Ristretto scalar field. This happens because the Edwards curve is defined over said field. As a method of creating equivalent points, is not dissimilar to how $X$, $Y$, and $Z$ projective coordinates can represent the same $P$ and $Q$ Edwards points for a given Edwards curve. The elements of the created prime-order group, ristretto255, are not curve points, they are simply represented by curve points. For computation understanding, it must be noted that not this prime-order group is not a subgroup of the curve and that there is an unequivocal distinction between the curve points and group elements. 
\section{Equations}
\subsection{Twisted Edwards and Montgomery Forms}
In order for a selected elliptic align with the goals defined in this paper, it needs to be both twist secure and Ristretto-ready. The Doppio curve has been chosen for the reasons highlighted above. \\\\
Which is defined as follows: \\\\
\noindent\fbox{%
    \parbox{\textwidth}{%
        \begin{itemize}
    \item Curve equation $$ -x^2+y^2=1-\frac{126296}{126297}x^2y^2 $$ Which is Twisted Edwards and used to implement Ristretto255.
    \item $a= -1$
    \item $d= \frac{126296}{126297}$
    \item $Basepoint: Y = \frac{3}{5}$\\
    \item Montgomery form equivalent: $$ y^2=x^3+505186x^2+x $$
    \item $A = 505186 $
    \item $Basepoint: X = 4$\\
    \item The curve order, G, is $$ 2^{252}+115924404605461509904689566245241897752 $$      
    \item Theorder of the subgroup, q, is $$ 2^{249}+15114490550575682688738086195780655237219 $$       \item The prime order of the Ristretto scalar field, l, is  $$ 2^{252} + 27742317777372353535851937790883648493 $$
    \item $Cofactor: h =\frac{G}{q} = 8$
\end{itemize}.
    }%
}


\subsection{Weierstrass Form}
\noindent\fbox{%
    \parbox{\textwidth}{%
\begin{itemize}
    \item Weierstrass form equivalent: \\  $$y^2=x^3+ax+b $$
    \item $a$ = 7237005577332262213973186563042994240857116359379907606001\\950828033483786813
    \item $b$ = 445582015604702849664
\end{itemize}
    }%
}\\\\
The computation of the Weierstrass form is made to prove point addition in the simplest possible form as this underlines all of the current elliptic curve operations. These initial operations on the field elements are inline, which is made to ensure the most efficient computation possible.\\\\

To better contextualise this curve to a use case within the Dusk Network, the bidding process can be used, as this connects several of the sections in this paper. The bidding process uses the arithmetic of the curve to perform operations, as well as the set inclusion principles to the properties of the bid. It is first necessary to show that a bid lies in the list of valid bids, i.e. is a subset of all valid bids. This is done by set membership to see if an element is part of the total set or by showing that the element is linear in N, where N is the size of the group. Then the necessary requirements for the bid are proven, which is making sure it hashes to the correct values. Following this, the bid is added to a vector of valid bids. A binary vector, which is a vector that compactly stores bits, is then created and this vector must be the same length as the vector of valid bids plus the created bid. In this binary format, a one is indicative of the position of your bid, and zero is indicative of the other bids.

\section{Field Elements}
For curve arithmetic to be performed, it is imperative to have a solid implementation. This allows for a basis on which the most primary operations can be carried out, the crucial nature of these operations stems from the ability to perform multiple cryptographic functions from only a few fundamental operations. \\\\
It is standard when implementing curves from their field elements, that point addition is the first function to be defined, as it is the foundation on which the rest of the operations stand. Point addition is simply adding points to one another along the elliptic curve.\\
The points which can be shown by $x$ and $y$, in Cartesian form, lie upon the elliptic curve and are all multiples of the generator point. Setting the prime field, over which the curve is defined, aside for a moment allows for more clear mental imagery of how point addition works. The image below depicts point addition on a standard elliptic curve, with good visual aids. The generator point, denoted as $G$, is the point from which the addition is begun until the next generator point is reached. This is done by taking a tangent to the Generator point and then reflecting it on the x-axis, because of the mirror symmetry properties$[7]$, which gives the next point. The image below provides the reader with a visual understanding of how the point addition can be performed:
\begin{center}
\includegraphics[width=8.5 cm]{images/ygncy.png}
\end{center}
Point addition varies from curve to curve and optimizations are continually performed whilst the field elements are created. The main rationale behind the need for optimization is to keep the operations time constant. The field elements are represented in bit terms, which are commonly converted to \texttt{u64} arrays. Unfortunately, the aforementioned formatting can lead to problems with the arithmetic in programming. These issues are often centred around over-spill, which occurs when making computations that have bit carrying. Such issues arises when using 32-byte arrays in addition, which impacts the overall performance as the operation leaves remainders due to the bit-carrying.\\\\
In order to avoid the issues mentioned above, radix representations of the field elements are utilized in order to avoid this bit-carrying as well as to eliminate any potential overflows created during addition, which makes the implementation more efficient. Every \textit{field element} has to be represented as an array of five \texttt{u64}'s (in a concrete radix representation), which enables the computation of the product in the form \texttt{u64} $\cdot$ \texttt{u64} = \texttt{u128}\footnote{Please note that the Zerocaf implementation is taking advantage of the Rust Programming Language support for 128-unsigned integer operations.}.
\\\\
To achieve this, the chosen radix is $2^{52}$, which is optimal for dealing with over-spill. An issue which arises from the use of bit terms is the computational speed of the field arithmetic operations. \\\\
In this case, it is known that the most expensive CPU operation is the integer division. In order to avoid the operation highlighted above, an implementation all of the curve arithmetics is combined with bit-shifting techniques$[8]$. Bit-shifting is simply done by moving a series of bits to the left or right to achieve greater efficiency in a mathematical operation. When dealing with radices, there is always a need to add an integer so that the another module can be achieved, this integer is what is used for bit-shifting. The selection of this integer is a simple arithmetic operation on the defined prime of the field. 
If we let $x$ be the remainder of the prime field, as shown below:
$$ l = 2^{252}+x $$
The value of the integer $x$ can be proven:
$$ p = 0\mod p$$  
$$ p = 2^{252}+x $$
$$ 2^{252}+x = 0\mod p $$
$$ 2^{252} = -x\mod p $$
The integer $x$ is then used in the calculations for radix $2^{52}$, so that a different module can be achieved. \\\\
From this point addition, many of the further operations are made elementary as they all work with the manipulation of points, in some mathematical relation.  
\section{Future work}
\section{Conclusion}
\section{Acknowledgements}


\newpage



\begin{thebibliography}{99}

\bibitem{c1} Stanford University, University College London and BlockStream, Benedikt Bünz, Johnathan Bootle, Dan Boneh, Andrew Polestra, Pieter Wuille and Greg Maxwell. Bulletproofs: Short Proofs for Confidential Transactions and More.\\ https://eprint.iacr.org/2017/1066.pdf
\bibitem{c2} Shafi Goldwasser, Silvio Micali, and Charles Rackoff. The knowledge complexity of interactive
proof-systems (extended abstract). In 17th Annual ACM Symposium on Theory of Computing
(STOC’85), pages 291–304, 1985."
\bibitem{c3} Pedersen T.P. (1992) Non-Interactive and Information-Theoretic Secure Verifiable Secret Sharing. In: Feigenbaum J. (eds) Advances in Cryptology — CRYPTO ’91. CRYPTO 1991. Lecture Notes in Computer Science, vol 576. Springer, Berlin, Heidelberg
\bibitem{c4} Isis Lovecruft and Henry de Valence. Ristretto. https://Ristretto.group/Ristretto.html
\bibitem{c5} Mike Hamburg : Deacaf. November, 2015. https://eprint.iacr.org/2015/673.pdf
\bibitem{c6} Feng Hao, Thales E-Security, Cambridge, UK https://eprint.iacr.org/2010/149.pdf
\bibitem{c7}Robert Dijkgraaf: Mirror Symmetry and Elliptic Curves, university of Amsterdam, November 15, 2002
\bibitem{c8} Tehcnological University of Visvesvaraya, Jnana Sangama https://www.academia.edu/8777556/

\end{thebibliography}




\end{document}